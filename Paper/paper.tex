\documentclass{llncs}
\usepackage{amsmath}
\usepackage{amssymb}
\usepackage{tikz}
\usepackage{graphicx,color}
\usepackage{multicol}
\usepackage{bussproofs}
\usepackage{float}
\usepackage{caption}
\usepackage{subcaption}

\usetikzlibrary{arrows,shapes,calc}

\newtheorem{lem}{Lemma}

\begin{document}

\tikzstyle{elem} = [circle]
\tikzstyle{line} = [draw,thick, -latex']

\title{Algebraic Principles for Concurrency Verification Tools}

\author{Alasdair Armstrong \and Victor B.~F.~Gomes \and Georg Struth}

\institute{Department of Computer Science, University of Sheffield, UK\\
\email{$\{$a.armstrong,v.gomes,g.struth$\}$@dcs.shef.ac.uk}}

\maketitle

\begin{abstract}
\end{abstract}

\pagestyle{plain}

%%%%%%%%%%%%%%%%%%%%%%%%%%%%%%%%%%%%%%%%%%%%%%%%%%%%%%%%%%%%%%%%%%%%%%%
\section{Introduction}

\cite{hoare_concurrent_2011}
\cite{kozen_completeness_1994}
\cite{kozen_kleene_1997}
\cite{armstrong_kleene_2013}
\cite{nipkow_isabelle/hol:_2002}

\newpage
%%%%%%%%%%%%%%%%%%%%%%%%%%%%%%%%%%%%%%%%%%%%%%%%%%%%%%%%%%%%%%%%%%%%%%%%%%%%%%
\section{Kleene Algebra}

\newpage
%%%%%%%%%%%%%%%%%%%%%%%%%%%%%%%%%%%%%%%%%%%%%%%%%%%%%%%%%%%%%%%%%%%%%%%%%%%%%%
\section{Rely/Guarantee}


$r, g \vdash \{p\}x\{q\} \longleftrightarrow p(r\|x)\le q \land x \le g$

\begin{prooftree}
\LeftLabel{Weakening}
\AxiomC{$r' \le r$}
\AxiomC{$g \le g'$}
\AxiomC{$p \le p'$}
\AxiomC{$r', g' \vdash \{p'\}x\{q'\}$}
\AxiomC{$q' \le q$}
\QuinaryInfC{$r, g \vdash \{p\}x\{q\}$}
\end{prooftree}

\begin{prooftree}
\LeftLabel{Sequential}
\AxiomC{$r, g \vdash \{p\}x\{q\}$}
\AxiomC{$r, g \vdash \{q\}y\{s\}$}
\BinaryInfC{$r, g \vdash \{q\}xy\{s\}$}
\end{prooftree}

\begin{prooftree}
\LeftLabel{Parallel}
\AxiomC{$r_1, g_2 \vdash \{p_1\}x\{q_1\}$}
\AxiomC{$g_1 \le r_2$}
\AxiomC{$r_1, g_2 \vdash \{p_2\}y\{q_2\}$}
\AxiomC{$g_2 \le r_1$}
\QuaternaryInfC{$r_1 \sqcap r_2, g_1 \| g_2 \vdash \{p_1 \sqcap q_2\}x\|y\{q_1 \sqcap q_2\}$}
\end{prooftree}

\begin{prooftree}
\LeftLabel{Choice}
\AxiomC{$r, g \vdash \{p\}x\{q\}$}
\AxiomC{$r, g \vdash \{p\}y\{q\}$}
\BinaryInfC{$r, g \vdash \{p\}x + y\{q\}$}
\end{prooftree}

\newpage
\section{Algebra for Rely/Guarantee}

Define a rely/guarantee algebra as a structure
$(K,RG,+,\sqcap,\cdot,\|,^\star,0,1)$, where $(K,+,\sqcap)$ is a
lattice, $(K,+,\cdot,\|,0,1)$ is a weak trioid and
$(K,+,\cdot,^\star,0,1)$ is a Kleene algebra. $RG$ is a distinguished
subset of relys/guarantees which satisfy the following axioms
\begin{align}
r\|r &\le r, \label{rg1}\\
r &\le r\|r', \label{rg2}\\
r\|xy &= (r\|x)(r\|y), \label{rg3}\\
r\|x^+ &\le (r\|x)^+ \label{rg4}.
\end{align}
By convention, we use $r$ and $g$ to refer to elements of $RG$, and
$x,y,z$ for arbitrary elements of $K$. Some elementary consequences of these rules are as follows
\begin{align*}
1 &\le r,\\
r^\star = rr &= r = r\|r,\\
r\|x^+ &= (r\|x)^+.
\end{align*}

Axioms (\ref{rg1}), (\ref{rg2}) and (\ref{rg3}) are independent. To
show this, we use Isabelle's \emph{Nitpick} counterexample generator
to construct models which violate each of these axioms, yet satisfy
all others. The models thus constructed for (\ref{rg1}) and
(\ref{rg3}) are shown in figures \ref{fig:rg1} and \ref{fig:rg3}
respectively. The model constructed for (\ref{rg2}) is just the two
element rely/guarantee algebra with $0$ and $1$.

\begin{figure}[H]
\centering
\begin{subfigure}{0.24\textwidth}
\begin{tikzpicture}[x=1.5cm,y=1.5cm,auto]
  \node (center) {};
  \node [elem] (r1) at (90:1) {$r_1$};
  \node [elem] (one) at (200:1) {$1$};
  \node [elem] (r2) at (20:1) {$r_2$};
  \node [elem] (zero) at (-90:1) {$0$};

  \path [line] (r1) -- (r2);
  \path [line] (r1) -- (one);
  \path [line] (r1) -- (zero);
  \path [line] (r2) -- (one);
  \path [line] (r2) -- (zero);
  \path [line] (one) -- (zero);
\end{tikzpicture}
\end{subfigure}
\begin{subfigure}{0.24\textwidth}
\begin{align*}
r_1 \| r_1 &= r_1\\
r_1 \| r_2 &= r_1\\
r_2 \| r_1 &= r_1\\
r_2 \| r_2 &= r_1
\end{align*}
\end{subfigure}
\begin{subfigure}{0.24\textwidth}
\begin{align*}
r_1r_1 &= r_1\\
r_1r_2 &= r_1\\
r_2r_1 &= r_1\\
r_2r_2 &= r_2
\end{align*}
\end{subfigure}
\begin{subfigure}{0.24\textwidth}
\begin{align*}
r_1^\star &= r_1\\
r_2^\star &= r_2
\end{align*}
\end{subfigure}
\caption{4 element counterexample for $r \in RG \implies r\|r \le r$}
\label{fig:rg1}
\end{figure}

Axiom (\ref{rg4}) can be derived from (\ref{rg3}) in all finite
models. This is because any finite $K$ is a weak left quantale with
respect to both sequential and parallel composition. In a quantale,
fixpoint fusion laws can be used to prove that
\begin{align*}
r\|x^+ = r\|(\mu y.\; x + xy) = \mu y.\; r\|x + r\|xy = (r\|x)^+.
\end{align*}
Thus it is impossible to construct a model which demonstrates
that (\ref{rg4}) is independent from (\ref{rg1}) -- (\ref{rg2}).

\begin{figure}
\centering
\begin{subfigure}{0.24\textwidth}
\begin{tikzpicture}[x=1.5cm,y=1.5cm,auto]
  \node (center) {};
  \node [elem] (r1) at (90:1) {$r_1$};
  \node [elem] (one) at (0:1) {$1$};
  \node [elem] (zero) at (-90:1) {$0$};

  \path [line] (r1) -- (one);
  \path [line] (r1) -- (zero);
  \path [line] (r1) -- (zero);
  \path [line] (one) -- (zero);
\end{tikzpicture}
\end{subfigure}
\begin{subfigure}{0.24\textwidth}
\begin{align*}
r_1 \| r_1 &= r_1\\
r_1 r_1 &= r_1\\
r_1^\star &= r_1
\end{align*}
\end{subfigure}
\caption{3 element counterexample for $r \in RG \implies r\|xy = (r\|x)(r\|y)$}
\label{fig:rg3}
\end{figure}

\newpage
%%%%%%%%%%%%%%%%%%%%%%%%%%%%%%%%%%%%%%%%%%%%%%%%%%%%%%%%%%%%%%%%%%%%%%%%%%%%%%
\section{Language Model}

\newpage
%%%%%%%%%%%%%%%%%%%%%%%%%%%%%%%%%%%%%%%%%%%%%%%%%%%%%%%%%%%%%%%%%%%%%%%%%%%%%%
\section{Example}

\newpage
%%%%%%%%%%%%%%%%%%%%%%%%%%%%%%%%%%%%%%%%%%%%%%%%%%%%%%%%%%%%%%%%%%%%%%%%%%%%%%
\section{Conclusion}

\bibliography{paper}{}
\bibliographystyle{plain}

\end{document}
